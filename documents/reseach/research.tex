\title{\bf Review of atleast thirteen photonics Student societies}
\author{Shawal Mbalire}
\newcommand{\abstractText}{\noindent
This research aims to explore the activities and characteristics of student photonics chapters at universities around the world.
The research includes a review of the programming and activities of 15 student photonics chapters.
The research also examines the challenges and successes of these chapters and investigates the potential benefits and drawbacks of different models for organizing and funding student photonics chapters.
The findings of this research provide valuable insights into the role of student photonics chapters in promoting the study and understanding of photonics among students and in supporting the development of their knowledge and skills in this field as well as formation of Makerere University Photonics Society(MUPhS).
}
\documentclass[10pt, a4paper, twocolumn]{article}
\usepackage{xurl}
\usepackage{titlesec}
\usepackage[super,comma,sort&compress]{natbib}
\usepackage{abstract}
\renewcommand{\abstractnamefont}{\normalfont\bfseries}
\renewcommand{\abstracttextfont}{\normalfont\small\itshape}
\usepackage{lipsum}

% Any configuration that should be done before the end of the preamble:
\titlespacing*{\section}{0pt}{0pt}{0pt}
\titlespacing*{\item}{0pt}{0pt}{0pt}
\usepackage{hyperref}
\hypersetup{colorlinks=true, urlcolor=blue, linkcolor=blue, citecolor=blue}

\begin{document}

    %%%%%%%%%%%%
    % Abstract %
    %%%%%%%%%%%%

    \twocolumn[
    \begin{@twocolumnfalse}
        \maketitle
        \begin{abstract}
        \abstractText
        \newline
        \end{abstract}
    \end{@twocolumnfalse}
    ]

    \section{PSG}
    The Photonics Society of Ghent (PSG), formerly known as the Ghent Optics Society, was formed in September 2016, when the SID-ME student branch and the SPIE student chapter of Ghent University joined hands.  In 2017, the OSA student chapter was founded and added to the organization.
    By combining the different student chapters, PSG is able to provide their members with the benefits of all three professional organizations.
    Rather than competing for members, these chapters now cooperate to organize more and bigger events.

    \section{Photobears}
    The joint student chapter of the SPIE (International Society for Optical Engineering), Optica (formerly, Optical Society of America), and the IEEE Photonics Society at the University of California, Berkeley.
    The club is operated by students who are dedicated to advancing the field of optics.
    Our members are affiliated with several departments on campus including Electrical Engineering, Physics, Chemistry, Mechanical Engineering, Astronomy, Vision Science and Optometry.
    We are committed to promoting the optics community on campus. To become involved in PhotoBears, attend a PhotoBears meeting or contact the current members.

    \section{Princeton Optica Student Chapter}
    The chapter works to raise awareness of optics and photonics through participation in events at Princeton University and in the local community through youth education outreach activities such as science fairs, lab tours and classroom demonstrations.
    Journal club meetings are hosted by graduate students to discuss current progress and trends in the field of optical science, with invited lectures offered by leading researchers in the field.

    \section{OSUM}
    The optical society at University of Michigan is a joint SPIE and OSA student chapter.
    It aims to promote the discipline of optical science and engineerng through research and discussion.

    \section{JIIT Optica}
    Established in September of 2017, the JIIT-Optica Student Chapter is Jaypee Institute of Information Technology’s only internationally recognised scientific body.
    A group of science enthusiasts that are hungry for impact.
    They conduct workshops on the latest skills in STEM, go for industry visits, compete in hackathons, do some research and attend international conferences.

    \section{University of Dhaka}
    Student Chapter of Optica (formerly named as OSA) at University of Dhaka was first founded in 2011 in the Department of Applied Physics and Communication Engineering (Now ‘Electrical And Electronic Engineering’). Later it was revived on 2021

    \section{B-PHOT}
    B-PHOT Student Chapter, affiliated with the Brussels Photonics Team, is a joint student chapter of the SPIE and Optica Vrije Universiteit Brussel Student Chapters. The Vrije Universiteit Brussels SPIE Student Chapter was founded in 2006-2007 as the first student chapter in Western Europe. The Vrije Universiteit Brussels Optica Student Chapter was founded in 2020-2021. By forming the B-PHOT Student Chapter, we continue to promote optics and photonics with a joint force. The mission of the Student Chapter is to bring together bachelor, master, and PhD students in photonics and offer them a platform to interact with each other, academics in photonics (post-docs, professors…), industrial photonics, and the international photonics societies.

    \section{PhE}
    The Photonics Society Endhoven (PhE) is a student community officially recognized as an Optica (formerly OSA) student chapter in March 2020.
    We are a group of enthusiastic and determined PhD students committed to the dissemination of optics and photonics in the city of Eindhoven.
    Our main goal is to promote enrollment in Optics and Photonics by creating opportunities for students to perform high-level scientific research in technical areas within Photonics.

    \section{OPSoc}
    Incorporating the University of Southampton Optica (formerly OSA) Student Chapter, the University of Southampton SPIE Student Chapter and the University of Southampton IEEE Photonics Society.The University of Southampton Optics and Photonics Society (OPSoc) consists of a group of postgraduate research students in the Faculty of Engineering and Physical Sciences, as well as related departments, at the University of Southampton.The students of the Optoelectronics Research Centre formed the society as a means to communicate their research to both internal and external audiences.

    \section{KYU Photonics}
    Focus on all aspects of Optoelectronic and Photonic materials, devices and Systems from Research and Development to Applications.

    \section{Glasgow Optics}
    We are students from Physics, Chemistry, Engineering and Biology who share academic interest in the field of Optics. Our main activity is the organization of seminars, conferences, career and outreach events (with the eventual social pub evening) to further expand the knowledge and networks of the members of our society. We are affiliated with world renowed organization in the field of Optics such as OSA and SPIE, which sponsor us and help us cover the costs of our events.

    \section{TPS}
    Taiwan Photonics Society was founded in 1980 and named as Optical Engineering Society of the Republic of China (ROCOES)” in order to promote the basic and application research of optics. Our major activities include academic conference, technical training, and collection and publication of optical information. And now there are more than 1,000 members in our society. We also have one journal entitled“Optical Engineering”and published quarterly.

    \section{Photonics at UCI}
    Optica and SPIE Student Chapter at the University of California, Irvine

    \section{Strathclyde Photonics Society}
    A Photonics Society Student Chapter at Strathclyde aimed at promoting interdisclipinary research and professional development.IEEE Photonics Student Chapter

    \section{OxOPS}
    The Oxford Optics and Photonics Student Society (OxOPS) is a student society supported by the Oxford Photonics Network. It provides a forum for research students and early stage researchers in photonics and maintains an active public presence all year round.Student Chapter of the Optical Society (OSA)

    \section*{Discussion}
    After reviewing the above 15 student chapters, we have identified the following key points that are common to all of them:\\
    \begin{enumerate}
        \item The student chapters are affiliated with the professional organizations, such as SPIE, Optica (formarly OSA), IEEE Photonics Society.
        \item The student chapters usually have a website and a Facebook page.
        \item They hold a number of events such as seminars, workshops, journal clubs, and conferences.
    \end{enumerate}
    As such, we have a rough idea of what events are usually held by student chapters.

    \section*{Conclusion}
    In this paper, we have reviewed the student chapters of the professional organizations, such as SPIE, Optica (formarly OSA), IEEE Photonics Society.
    We have embarked on a journey to explore the student chapters of the professional organizations looking for ways to form chapters with them.

    %--------------References---------------%
    \nocite{*}
    \bibliographystyle{plain}
    \bibliography{test}

\end{document}
